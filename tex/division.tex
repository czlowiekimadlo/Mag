% *** Uwaga *** Tekst w tym pliku nie ma specjalnie sensu i służy za twz. wypełniacz ***
\documentclass[skorowidz,autorrok,backref,xodstep]{wmimgr}

\usepackage{listings}
\usepackage{color}
\usepackage{alltt}
\usepackage{floatrow}
\usepackage{hyphenat}
\usepackage{url}

\newcommand\dbr{\discretionary{}{}{}}


% Opcjonalnie identyfikator dokumentu (drukowany tylko z włączoną opcją `brudnopis'):
\nrwersji {0.1}

% Dane autora(ów):
\author   {Wojciech Chojnacki}
\nralbumu {369971}
\email    {s367791@wmi.amu.edu.pl}

% Tytuł pracy:
\title    {Hierarchiczny model podziału obiektów}

% Kierunek, tj. katedra/instytut promotora:
\kierunek {Informatyka}

% Rok obrony:
\date     {2013}

% Jeżeli nie podano miejsca zostanie wpisany `Sopot'
\miejsce {Poznań}

% Tytuł naukowy, imię i nazwisko promotora:
\opiekun  {dr Wojciech Kowalewski}

%
% Miejsce na deklaracje własnych poleceń:
\newcommand{\filename}[1]{\texttt{#1}}

% Cytowanie przez numer (standard):
%\bibliographystyle{plain}
%
% Jeżeli cytowanie autor-rok to np.:
\bibliographystyle{papalike}
%
% Inne sposoby
%\bibliographystyle{abbrv} %% standard
%\bibliographystyle{acm} %% ACM transactions...
%\bibliographystyle{elsart-harv} %% dziwaczny %%

%%% zakomentuj \iffalse ... \fi (ostatnie, zaznaczone //pdfscreen) jeżeli chcesz włączyć pakiet pdfscreen:
\def\SITI{SI/TI} %%%
\def\ISTI{SI/TI} %%%
\def\UTAUT{UTAUT} %%%

\begin{document}

%%
%%\nocite{beebe,p.perl} %% dołącza niecytowane
\nocite{*} %% ** dołącza wszystko **


% Tytuł/spis treści
\maketitle
%
% Wstęp
\introduction

Wstęp.

\chapter{Cel Pracy}

Zastosowania.

Obecne rozwiązania.

\chapter{Diagram Voronoi}

Diagramy.

\chapter{Triangulacja Delaunay}

Triangulacje.

\chapter{Przejście z Przestrzeni 2D do 3D}

Przejście.

\chapter{Implementacja i Wyniki}

Kod.

\chapter{Dalsze Prace}

Optymalizacja implementacji.

Podziały przez kotki/kulki.

Przejście do płynu.

Warunki zmiany algorytmu.

Dystrybucja punktów.

%\include bib.tex
\bibliography{division}

%
% Spis tabel (jeżeli jest potrzebny):
%\listoftables
%
% Spis rysunków (jeżeli jest potrzebny):
\listoffigures

%
% Skorowidz (opcjonalnie)
% \printindex

\end{document}
